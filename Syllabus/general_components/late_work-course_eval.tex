% Options for packages loaded elsewhere
\PassOptionsToPackage{unicode}{hyperref}
\PassOptionsToPackage{hyphens}{url}
%
\documentclass[
]{article}
\usepackage{amsmath,amssymb}
\usepackage{lmodern}
\usepackage{iftex}
\ifPDFTeX
  \usepackage[T1]{fontenc}
  \usepackage[utf8]{inputenc}
  \usepackage{textcomp} % provide euro and other symbols
\else % if luatex or xetex
  \usepackage{unicode-math}
  \defaultfontfeatures{Scale=MatchLowercase}
  \defaultfontfeatures[\rmfamily]{Ligatures=TeX,Scale=1}
\fi
% Use upquote if available, for straight quotes in verbatim environments
\IfFileExists{upquote.sty}{\usepackage{upquote}}{}
\IfFileExists{microtype.sty}{% use microtype if available
  \usepackage[]{microtype}
  \UseMicrotypeSet[protrusion]{basicmath} % disable protrusion for tt fonts
}{}
\makeatletter
\@ifundefined{KOMAClassName}{% if non-KOMA class
  \IfFileExists{parskip.sty}{%
    \usepackage{parskip}
  }{% else
    \setlength{\parindent}{0pt}
    \setlength{\parskip}{6pt plus 2pt minus 1pt}}
}{% if KOMA class
  \KOMAoptions{parskip=half}}
\makeatother
\usepackage{xcolor}
\IfFileExists{xurl.sty}{\usepackage{xurl}}{} % add URL line breaks if available
\IfFileExists{bookmark.sty}{\usepackage{bookmark}}{\usepackage{hyperref}}
\hypersetup{
  hidelinks,
  pdfcreator={LaTeX via pandoc}}
\urlstyle{same} % disable monospaced font for URLs
\usepackage[margin=1in]{geometry}
\usepackage{graphicx}
\makeatletter
\def\maxwidth{\ifdim\Gin@nat@width>\linewidth\linewidth\else\Gin@nat@width\fi}
\def\maxheight{\ifdim\Gin@nat@height>\textheight\textheight\else\Gin@nat@height\fi}
\makeatother
% Scale images if necessary, so that they will not overflow the page
% margins by default, and it is still possible to overwrite the defaults
% using explicit options in \includegraphics[width, height, ...]{}
\setkeys{Gin}{width=\maxwidth,height=\maxheight,keepaspectratio}
% Set default figure placement to htbp
\makeatletter
\def\fps@figure{htbp}
\makeatother
\setlength{\emergencystretch}{3em} % prevent overfull lines
\providecommand{\tightlist}{%
  \setlength{\itemsep}{0pt}\setlength{\parskip}{0pt}}
\setcounter{secnumdepth}{-\maxdimen} % remove section numbering
\ifLuaTeX
  \usepackage{selnolig}  % disable illegal ligatures
\fi

\author{}
\date{\vspace{-2.5em}}

\begin{document}

\hypertarget{late-work-and-student-emergencies}{%
\subsection{Late Work and Student
Emergencies}\label{late-work-and-student-emergencies}}

Generally, late papers, past due participation in Perusall or
extra-credit work will \emph{not} be accepted in this course. However,
there are some important exceptions to this rule. Whenever unexpected
life challenges/emergencies or an illness interrupts your academic work,
other measures will apply. If you are ill or face an emergency, you are
asked to please notify the instructor immediately (via email or in
person), and provide documentation of what has occurred to justify
making-up your work or turning it in beyond it due date. Moreover, given
the current COVID-19 pandemic, I am aware that your situation can change
dramatically unexpectedly.
\textcolor{red}{If you are facing any personal, medical, financial or other challenges that impact your academic work, please inform me right away and we will work out a solution to your problem. I will do my best to accommodate your situation and allow you to complete your work at a later time if you missed a deadline.}
\textbf{\textcolor{red}{However, the longer you wait, the less likely I am to accept a late assignment. Furthermore, I reserve the right to refuse any late work, but keeping the lines of communication open is your best bet if you fall a little behind on an assignment.}}

\faAsterisk Please also note that if you foresee being absent due to a
conflict with religious observance(s), you must notify the instructor of
your prospective absence during the first week of classes. If you have
any questions, do not hesitate to discuss these issues with the
instructor in person or via BlackBoard.

\end{document}
