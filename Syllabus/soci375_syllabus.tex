\documentclass[11pt,]{article}
\usepackage[margin=1in]{geometry}
\newcommand*{\authorfont}{\fontfamily{phv}\selectfont}
\usepackage[]{mathpazo}
\usepackage{multicol}
\usepackage{abstract}
\renewcommand{\abstractname}{}    % clear the title
\renewcommand{\absnamepos}{empty} % originally center
\newcommand{\blankline}{\quad\pagebreak[2]}

\providecommand{\tightlist}{%
  \setlength{\itemsep}{0pt}\setlength{\parskip}{0pt}} 
\usepackage{longtable,booktabs}
      
\usepackage{parskip}
\usepackage{titlesec}
\titlespacing\section{0pt}{12pt plus 4pt minus 2pt}{6pt plus 2pt minus 2pt}
\titlespacing\subsection{0pt}{12pt plus 4pt minus 2pt}{6pt plus 2pt minus 2pt}

\titleformat*{\subsubsection}{\normalsize\itshape}

\usepackage{titling}
\setlength{\droptitle}{-.15cm}

%\setlength{\parindent}{0pt}
%\setlength{\parskip}{6pt plus 2pt minus 1pt}
%\setlength{\emergencystretch}{3em}  % prevent overfull lines 

\usepackage[T1]{fontenc}
\usepackage[utf8]{inputenc}

\usepackage{fancyhdr}
\pagestyle{fancy}
\usepackage{lastpage}
\renewcommand{\headrulewidth}{0.3pt}
\renewcommand{\footrulewidth}{0.0pt} 
\lhead{}
\chead{}
\rhead{\footnotesize SOCI 375: Global Problems -- Fall 2022}
\lfoot{}
\cfoot{\small \thepage /\pageref*{LastPage}}
\rfoot{}

\fancypagestyle{firststyle}
{
\renewcommand{\headrulewidth}{0pt}%
   \fancyhf{}
   \fancyfoot[C]{\small \thepage/\pageref*{LastPage}}
}

%\def\labelitemi{--}
%\usepackage{enumitem}
%\setitemize[0]{leftmargin=25pt}
%\setenumerate[0]{leftmargin=25pt}




\makeatletter
\@ifpackageloaded{hyperref}{}{%
\ifxetex
  \usepackage[setpagesize=false, % page size defined by xetex
              unicode=false, % unicode breaks when used with xetex
              xetex]{hyperref}
\else
  \usepackage[unicode=true]{hyperref}
\fi
}
\@ifpackageloaded{color}{
    \PassOptionsToPackage{usenames,dvipsnames}{color}
}{%
    \usepackage[usenames,dvipsnames]{color}
}
\makeatother
\hypersetup{breaklinks=true,
            bookmarks=true,
            pdfauthor={},
             pdfkeywords = {},  
            pdftitle={SOCI 375: Global Problems},
            colorlinks=true,
            citecolor=blue,
            urlcolor=blue,
            linkcolor=magenta,
            pdfborder={0 0 0}}
\urlstyle{same}  % don't use monospace font for urls


\setcounter{secnumdepth}{0}





\usepackage{setspace}

\title{SOCI 375: Global Problems}
\date{Fall 2022}


\begin{document}  

		\maketitle
		
	
		\thispagestyle{firststyle}

%	\thispagestyle{empty}
\textbf{\underline{Instructor}}
\begin{multicols}{2}
  \end{multicols}
	
\noindent \begin{tabular*}{\textwidth}{ @{\extracolsep{\fill}} lr @{\extracolsep{\fill}}}
\textbf{\underline{Class information}}\\
  Classroom: Online\\
  Class Hours: 1:00 p.m. - 2:15 p.m. MW\\
    \\
	\end{tabular*}\\


\vspace{2mm}


\hypertarget{indigenous-land-statement}{%
\section{Indigenous Land Statement}\label{indigenous-land-statement}}

\begin{quote}
\begin{quote}
\emph{Please be aware that the land on which our campus resides occupies
the territory of the Mechoopda people. Without them, we would not have
access to this campus or our education.}
\end{quote}
\end{quote}

\hypertarget{course-status}{%
\section{Course Status}\label{course-status}}

We explore historic, economic, political, ecological and social changes
that have created social problems worldwide. We examine a broad range of
global problems and current global developments through a macro
sociological lens. 3 hours lecture.

\hypertarget{course-description}{%
\section{Course Description}\label{course-description}}

\emph{Why study social problems through a global perspective?} Over the
past century, societies across the world have become increasingly
interconnected and, as a result, many social problems have a global
dimension to them. Through a global perspective, we can understand and
comprehend how \emph{globalization}-a process of increasing
international economic, political and cultural connection-generates
contradictory social outcomes (negative and positive) across the globe.
For instance, alongside a global reduction in poverty and overall
progress in life expectancy, the threat of climate change looms large
and the growing concentration of wealth amongst a small yet powerful
elite continues to come at the expense of working and lower-income class
groups. Therefore, it is important to ask: \emph{how is this possible?}

In this course, then, the main learning goal is to construct a more
precise understanding of these multifaceted global \emph{social}
problems, one that can be the basis of a critical evaluation of the past
century of global transformation. While there are various co-existing
and valid perspectives on globalization, our perspective on
globalization will be grounded on the tradition of critical political
economy and economic development. Critical political economy---a
tradition that includes Marxist theory, world- systems analysis,
critical globalization studies, feminist political economy, among
others--- grounds globalization in the processes of capitalist
expansion. While globalization is a complex and multi-dimensional
phenomenon that includes cultural interconnectedness and geopolitical
conflict, in this course we will examine the extent to which these
processes and social problems relate to the global reach of capitalist
production. This critical perspective is useful because much of our
daily lives depends on how the larger, global capitalist economy
functions and operates.

Beyond the core issue of globalization as the expansion of capitalism,
we will look at different specific instances or aspects of
globalization. Thus, the course is divided into one introduction and
three parts. The first two weeks---entitled \textbf{Introduction:
Theoretical and Historical Foundations}---will focus on clarifying the
different meanings of globalization and on identifying some of the most
salient socio-economic trends of the globalization era. \textbf{Part I},
entitled \textbf{Global Processes: Accumulation and Trade}, will unpack
some of the core processes that characterize contemporary globalization,
namely (1) the transformation of production processes from Fordist to
neoliberal strategies and (2) the expansion of free trade (in the sense
of free trade ideologies and actual trade in goods and services). After
that, we will move on to \textbf{Part II}, which centers on
\textbf{Dimensions of Globalization}. This second part of the course
will examine some important aspects of globalization---issues of labor,
gender, environment, and health will be discussed during the weeks that
composed the second part of the course. Lastly, \textbf{Part III
(Concrete Analysis of Concrete Situations)} will be about concrete
regional examples and resistance. First, we will evaluate the place of
Latin America and China in the globalization era. While we will be
looking at resistance all throughout the semester, in Part III we will
also examine some concrete issues and dilemmas surrounding resistance to
globalization. This look at resistance will function as a conclusion to
the course. By then of the semester we should have some general picture
of capitalist globalization, its origins, its contradictions and the
prospects for the future.

\hypertarget{course-learning-objectives-clo}{%
\subsection{Course Learning Objectives
(CLO):}\label{course-learning-objectives-clo}}

\begin{enumerate}
\def\labelenumi{\arabic{enumi}.}
\tightlist
\item
  Identify and describe the socio-economic and historical links that
  connect individual, local and national issues to global socio-economic
  transformations.
\item
  Explain core concepts, main arguments and relevant empirical
  data/examples used to discuss globalization and global problems in
  lectures and assigned readings.
\item
  Recognize the multiple dimensions and complex dynamics that shape
  globalization.
\item
  Analyze in writing what a selection of social scientists/historians
  argue regarding globalization. Simultaneously, you will learn to add
  some critical insight vis-à-vis these arguments.
\item
  Apply the theories learned to historical and/or contemporary events.
\item
  \textbf{The main goal} is to evaluate---in a nuanced and critical
  manner---contemporary globalization and the arguments made in favor or
  against it.
\end{enumerate}

\hypertarget{sociology-program-student-learning-objectives}{%
\subsection{Sociology Program Student Learning
Objectives:}\label{sociology-program-student-learning-objectives}}

\begin{itemize}
\tightlist
\item
  Demonstrate critical thinking through verbal and written
  communication.
\item
  Demonstrate the ability to design and evaluate quantitative and
  qualitative research.
\item
  Apply, critically interpret and synthesize sociological theory.
\item
  Display an understanding of and appreciation for cultural diversity.
\item
  Illustrate an understanding of the processes and implications of
  globalization.
\item
  Exhibit an understanding of the structural and interpersonal basis of
  social inequality.
\item
  Recognize the relationship between personal agency, social
  responsibility, and social change (sociological imagination).
\item
  Demonstrate proficiency in the use of technology.
\end{itemize}

\hypertarget{important-dates}{%
\subsection{Important Dates}\label{important-dates}}

\begin{itemize}
\tightlist
\item
  Sept.~2nd - Last day to drop without professor's approval\\
\item
  Oct.~24th - Last day to drop and receive financial credit for refund\\
\item
  Nov.~10th - Last day to drop or withdraw\\
\item
  See the
  \href{https://www.csuchico.edu/apss/calendar/aca-cal-2022-23.shtml}{2022-2023
  Academic Calendar} for more information.
\end{itemize}

\hypertarget{required-readings}{%
\section{Required Readings}\label{required-readings}}

\begin{itemize}
\tightlist
\item
  There is no required textbook for this course. All primary readings
  will be available electronically on the Canvas course page and on
  Perusall (more details about \protect\hyperlink{perusall}{Perusall}
  below). These readings will only be available to students enrolled in
  the course, and are to be used only for educational purposes.
\end{itemize}

\hypertarget{attendance}{%
\section{Attendance Policy}\label{attendance}}

\textbf{Attendance and ongoing participation in sociology courses is
mandatory in all sociology classes and is essential to your success in
this course.} Only documented emergencies will constitute excused
absences or assignments. Documentation will only be accepted within one
week of your return to class. Please do not contact me about
non-emergency absences.
\textcolor{red}{\bf{You are allowed up to three (3) unexcused absences and anything more than that will affect your final grade. If you expect to miss class because of a religious observance, you will need to notify me beforehand. Finally, if you miss class, it is your responsibility to catch up.}}

\hypertarget{dropped-from-the-course}{%
\subsection{Dropped From The Course}\label{dropped-from-the-course}}

In accordance with the college drop policy, students who do not access
the class web site or submit the designated ``first assignments'' by the
due date may be dropped as a no show. Students who fail to turn in the
weekly assignments and/or participate in the discussion boards will be
considered ``absent'' and may be dropped from the class for ``excessive
absence.'' Students are responsible for officially dropping the class or
they will receive a letter grade based on their performance.

These are two reasons that a student may be dropped from the course:

\begin{itemize}
\item
  \textbf{No Show Drop:} You must complete a designated discussion post,
  quiz, or practice activity by Friday of the first week of class or you
  may be dropped as a no-show. Students who do not complete \textbf{any}
  work in the first two weeks will be dropped.
\item
  \textbf{Excessive Absence Drop:} If more than two consecutive weeks of
  non-participation is observed by the professor the student may be
  dropped. If you do not participate in each week's participation
  activities (during class, online activities, or Perusall) and complete
  at least one quiz or practice activity, you will be considered
  ``non-participating'' and subject to being dropped from the course.
\end{itemize}

\textcolor{red}{\bf{All weekly lectures must be watched either live or within 48 hours.}}

\hypertarget{determination-of-final-grade}{%
\section{Determination of Final
Grade}\label{determination-of-final-grade}}

Table \ref{tab:grade_scale} displays the grading scale that is used to
determine a letter grade.

\renewcommand{\arraystretch}{1.5}

\begin{table}[!h]

\caption{\label{tab:grade_scale}Grading Scale}
\centering
\fontsize{10}{12}\selectfont
\begin{tabular}[t]{l|c|l}
\hline
\textbf{Grade} & \textbf{Range} & \textbf{Quality}\\
\hline
A & 93 - 100\% & Excellent Work\\
\hline
A- & 90 - 92\% & Nearly Excellent Work\\
\hline
B+ & 87 - 89\% & Very Good Work\\
\hline
B & 83 - 86\% & Good Work\\
\hline
B- & 80 - 82\% & Mostly Good Work\\
\hline
C+ & 77 - 79\% & Above Average Work\\
\hline
C & 73 - 76\% & Average Work\\
\hline
C- & 70 - 72\% & Mostly Average Work\\
\hline
D+ & 67 - 69\% & Below Average Work\\
\hline
D & 60 - 66\% & Poor Work\\
\hline
F & < 60\% & Failing Work\\
\hline
\end{tabular}
\end{table}

Table \ref{tab:weight_table} displays how your final grade will be
weighed according to the following components:

\renewcommand{\arraystretch}{1.5}

\begin{table}[!h]

\caption{\label{tab:weight_table}Course Requirements and Grading Policy}
\centering
\fontsize{10}{12}\selectfont
\begin{tabular}[t]{lc}
\toprule
\textbf{Item} & \textbf{\% of Final Grade}\\
\midrule
Two Exams & 30 \%\\
Two Critical Interpretation Essays & 20 \%\\
Five Quizzes & 15 \%\\
Paired Presentation & 15 \%\\
Weekly Perusall Participation & 10 \%\\
Attendance \& Participation & 10 \%\\
\textbf{TOTAL} & \textbf{100 \%}\\
\midrule
\bottomrule
\end{tabular}
\end{table}

\textbf{Note regarding class workload:} This is a three (3) credit hour
class. You should expect to spend two hours outside of class for every
one hour in class. This time outside of class will be spent on reading,
assignments, and studying. This means you should plan on devoting six
hours of outside time per week to this course.

\hypertarget{two-exams-30-of-final-grade-15-each}{%
\subsection{Two Exams (30\% of Final Grade; 15\%
each)}\label{two-exams-30-of-final-grade-15-each}}

Each exam can be comprised of multiple choice, true/false, definitions,
short answer, and essay questions. \textbf{You will be allowed 1 page
worth of notes.} The mid-term will cover the material covered from week
1 to 7. The final exam will cover the rest of the semester. Detailed
instructions for the exam will be provided via Canvas as each exam
approaches.

\hypertarget{writing_assignments}{%
\subsection{Two (2) Critical Interpretation Essay Papers (20\% of Final
Grade; 10\% each)}\label{writing_assignments}}

The essays' goal is to demonstrate analytical thinking and a nuanced
understanding of the material discussed in class. You will write
critical comments on the assigned readings combined with illustrations
from contemporary or historical examples. You will use both theoretical
notions and empirical evidence provided in the readings in order to
support your argument. Detailed guidelines and prompt questions will be
provided for the essay through Canvas. The first essay is due by
\textbf{Sunday September 25, by 11:59 p.m.}, and the second by
\textbf{Friday October 28, by 11:59 p.m.}. Additionally, you have the
opportunity to re-write your lowest-scored critical interpretation essay
for a higher grade. If you decide to rewrite one or two lower-scored
papers, these will be due in class on \textbf{Tuesday, December 13, 2022
(our ``final exam'' due date) no later than 11:59 p.m.}

\textbf{Each paper must be turned in no later than 11:59 p.m. (Pacific
Standard Time) on the day that it is due.}

\hypertarget{guidelines-for-papers}{%
\subsubsection{Guidelines for papers}\label{guidelines-for-papers}}

Papers have a word limit, approximately 1200 words (roughly 5
double-spaced pages); must be typed with 12-point font, 1-inch margins,
and Times New Roman font (11-point Arial font is also appropriate).

Please also follow these document design conventions:

\begin{itemize}
\tightlist
\item
  Your name, course number, and paper's title at the top of the first
  page. \textbf{This information does not count toward the word count.}
\item
  You do not need a separate title page or a running head.
\item
  Page numbers on all pages.
\item
  APA-style in-text citations and references list in APA format.
  \textbf{Citations do not count toward word length requirements.}
\item
  Proofread as appropriate for the document's stage in the writing
  process. All final versions must be proofread for mechanical errors
  (spelling, grammar, punctuation). All papers in this course must
  follow conventions of written academic English.
\item
  All assignments must be submitted via Canvas. Papers will be accepted
  in .doc, .docx, .rtf, and .pdf formats.
\end{itemize}

\textcolor{red}{\bf{No other format will be accepted.}}

\hypertarget{tips-and-tricks-for-good-writing}{%
\subsubsection{Tips and tricks for good
writing}\label{tips-and-tricks-for-good-writing}}

Use a citation management software to organize your papers and create
bibiliographies. My personal preference is
\href{https://www.zotero.org/}{Zotero} because it is free, open source,
and works across multiple operating systems. For information about other
citation management software that is available to CSU Chico students,
visit the Meriam Library
\href{https://libguides.csuchico.edu/c.php?g=432300\&p=2948649}{``Sample
Papers and Citation Tools'' page}.

Write and then revise. Ask for feedback from peers or wait until you
have some distance from your writing and then revise.

\hypertarget{quizzes}{%
\subsection{Five Quizzes (15\% of Final Grade; 3\%
each)}\label{quizzes}}

Five (5) quizzes are required in this class. The quizzes will cover
theories, ideas, and concepts from the course readings and lectures. The
syllabus quiz will be posted on Canvas by the end of the first class and
must be completed by \textbf{11:59 p.m. on Friday September 2nd}. The
other four quizzes will cover theories, ideas, and concepts from the
course readings and lectures. \textbf{These four required reading
quizzes will be posted electronically on Wednesdays at 2:30 p.m. and
must be completed by 11:59 p.m. on Thursday.} Each quiz will contain
several multiple choice questions based on course readings and lecture
material. These quizzes are designed to help you absorb the course
readings and lectures, and to provide you with the opportunity to assess
your mastery of the theories we are covering in the course. You will be
allotted between 45-60 minutes to complete a quiz. More details will be
posted on Canvas.

\textbf{Important Note: Please keep in mind that your lowest quiz score
will not be counted when computing final grades. Instead, your highest
quiz score will be counted twice in replacement of your lowest quiz
score. All quizzes will be posted on Canvas and you will be provided
advanced notice.}

\hypertarget{paired-presentation-15-of-final-grade}{%
\subsection{Paired Presentation (15 \% of Final
Grade)}\label{paired-presentation-15-of-final-grade}}

Students will work in pairs examining one dimension of globalization---
inequality trends, labor processes, gender dynamics, environmental
injustice, etc.--- and prepare a live video presentation. The evaluation
of the presentation/video will be based on the following criteria:
rapport with audience; voice, projection, and audibility; clarity in the
expression of ideas; organization; persuasive use of evidence; and
ability to respond to questions and to accept valid criticisms. I will
post specific guidelines for your oral presentations on the course's
Canvas site. \textbf{The paired presentation/video is due on Friday
December 9, by 11:59 p.m.}

\hypertarget{perusall}{%
\subsection{Weekly Perusall Participation (10\% of Final
Grade)}\label{perusall}}

For reading the primary texts (and not the textbook) we will use
\href{https://support.perusall.com/hc/en-us}{Perusall}, a social
annotation tool that allows students and their instructors to
collaboratively markup .pdf documents. Instead of only reading a
document and discussing it, Perusall brings discussion to text.

To start, register for a free Perusall account and enter the course
code: {[}Insert Code Here{]}

At the beginning of the semester, you will be assigned to groups in
Perusall, each comprised of around 8-10 students. After week 2, you will
be randomly placed into these groups and participate in asynchronous
group reading and discussion, covering any major theme(s)s or issue(s)
from the reading and lecture. You will each read and discuss within your
group forum. This means that you must not only do the reading and
annotate on your own, but you must ask and respond to one or more
questions within your group that are related to that week's readings
assignment(s) and lecture. Thus, you must interact in your group forum
with at least two (2) of your groupmate's posts. Your contribution and
annotations will be graded on a scale of 1 -- 3, with higher scores
indicating higher quality annotations and responses. \textbf{Your
annotations and responses to two or more groupmate's annotations and
questions are due each Wednesday by 11:59 p.m. Discussion posts and
annotations will be given credit only if they are substantive and follow
all instructions.}

For reading assignments in Perusall, you will be evaluated based on the
following five criteria:

\begin{enumerate}
\def\labelenumi{\arabic{enumi}.}
\tightlist
\item
  Quality: The more thoughtful the responses, the higher the score.
\item
  Quantity: Each student must submit at least seven (7) annotations in
  each reading document.
\item
  Timeliness: You must complete the reading assignment by
  \textbf{Wednesday 11:59 p.m.}
\item
  Distribution: Are annotations made throughout the document? If most
  annotations are made in a single area, it indicates the student did
  not engage with the entire document. If annotations are distributed
  throughout the entire document, it indicates the student did engage
  with the reading in entirety.
\item
  Asking/Answering Questions: Ask and respond to each other's comments
  and questions in their group forums.
\end{enumerate}

\textbf{Common Question: What if I don't have any questions or
comments?} Elaborate on a specific topic, connect to personal experience
or perspective, extension to another topic or perspective, paraphrase
part of the text.

\hypertarget{attendance-and-participation-10-of-final-grade}{%
\subsection{Attendance and Participation (10\% of Final
Grade)}\label{attendance-and-participation-10-of-final-grade}}

For this course, the participation grade will include various things.
First, participation requires attendance to
\colorbox{GreenYellow}{all Zoom lectures which will take place Mondays and Wednesdays 8:00 to 9:30 am.}
The participation grade will also include any in-class activities and
group exercises that we may do during Zoom meetings or immediately after
a meeting. That is, throughout the semester I will assign short
activities---these will take place either during class or to be
completed immediately after a lecture or as a discussion post---that
will count toward your overall participation grade. Note that
\textbf{Discussion Board Posts} will be a 33\% your participation
grade---we will have a number of them during the semester.

In addition to completing all of the assigned reading before each
designated day of the week (see reading schedule below and make sure you
attend class for the ongoing reading deadlines), you must critically
reflect on what is interesting/troubling/useful about what you have
read. You are expected to do this work on your own. Be prepared to offer
your questions, insights, and critique either via communication with me,
writing assignments and/or in conversation with other students.

\textbf{REMEMBER! Attendance is mandatory in all sociology classes (see
\protect\hyperlink{attendance}{Attendance Policy section} above). It is
also crucial to doing well in this class. In the context of this class,
attendance means COMPLETING ALL OF THE ASSIGNMENTS, QUIZZES, INDIVIDUAL
WORK, etc.}

\hypertarget{extra-credit-film-reviews-analyses-maximum-of-10-points5-points-per-analysis}{%
\subsection{Extra-Credit (Film Reviews \& Analyses) {[}Maximum of 10
Points/5 Points Per
Analysis{]}}\label{extra-credit-film-reviews-analyses-maximum-of-10-points5-points-per-analysis}}

Students will have the opportunity to earn extra credit in this course.
If you would like to earn up to 10 extra points maximum toward your
final course grade, you can write two 1-page (single-spaced) analytical
reviews and critiques of two films. The film analyses need to provide a
\emph{critical} sociological review and analysis that connects to some
of the main themes of the course. A list of suggested films/
documentaries will be posted on Canvas by the end of the third week of
classes along with the guidelines for this assignment. \textbf{These
extra credit critiques should be concise and well-organized and argued;
the film analyses can be submitted at any time during the semester, but
no later than Friday December 9 by 11:59 p.m.}

\hypertarget{late-work-and-student-emergencies}{%
\subsection{Late Work and Student
Emergencies}\label{late-work-and-student-emergencies}}

Generally, late papers, past due participation in Perusall or
extra-credit work will \emph{not} be accepted in this course. However,
there are some important exceptions to this rule. Whenever unexpected
life challenges/emergencies or an illness interrupts your academic work,
other measures will apply. If you are ill or face an emergency, you are
asked to please notify the professor immediately (via Canvas email or in
person), and provide documentation of what has occurred to justify
making-up your work or turning it in beyond it due date. Moreover, given
the current COVID-19 pandemic, I am aware that your situation can change
dramatically unexpectedly.
\textbf{\textcolor{red}{If you are facing any personal, medical, financial or other challenges that impact your academic work, please inform me right away and we will work out a solution to your problem. I will do my best to accommodate your situation and allow you to complete your work at a later time if you missed a deadline.}
\textcolor{red}{However, the longer you wait, the less likely I am to accept a late assignment. Furthermore, I reserve the right to refuse any late work thus, keeping the lines of communication open is your best bet if you fall a little behind on an assignment.}}

The professor's policy regarding late work is to give any student who is
ill or has faced a \textbf{documented medical/personal emergency} the
opportunity to make up their academic work \textbf{within one week of
the absence.} Therefore, it is imperative that you discuss with the
professor any emergencies and/or problems in order to assist you in
completing your coursework.
\textcolor{red}{\bf{Please also note that if you foresee being absent due to a conflict with religious observance(s), you must notify the professor of your prospective absence during the first week of classes.}}

If you have any questions, do not hesitate to discuss these issues with
the professor in person or via Canvas.

\hypertarget{getting-help-and-communicating-with-your-professor}{%
\section{Getting Help And Communicating With Your
Professor}\label{getting-help-and-communicating-with-your-professor}}

Remember, your professor is here to help you gain a rich understanding
of the material. So, it is important to be connected to your professor
and seek help whenever you need it. However, it is primarily your
responsibility to seek help when you need it. Use the following methods
to get in touch and/or ask questions.

\hypertarget{communication-policy}{%
\subsection{Communication policy}\label{communication-policy}}

When contacting your about the course, please send message through
``Inbox'' in Canvas. I check messages three times per day, Monday thru
Friday, between 9:00 AM and 6:00 PM. Please allow up to 24-48 hours for
a response during the week. If you contact on the weekend, expect a
reply the following Monday.

\begin{itemize}
\item
  Requesting help: If you miss a class meeting, you should\ldots{}

  \begin{enumerate}
  \def\labelenumi{\arabic{enumi}.}
  \tightlist
  \item
    First, look on the course website for material you missed.
  \item
    Second, if you need help understanding the material, you should
    contact your classmate (either via direct email or Canvas discussion
    board).
  \item
    Third, if you still find it difficult, find some time to chat with
    me during office hours.
  \end{enumerate}
\item
  Office Hours

  \begin{quote}
  These open office hours are first come, first served, so you may need
  to wait in the Zoom waiting room if I am speaking with other students.
  \end{quote}

  \begin{quote}
  \emph{Need a conversation starter?} Ask me about movies, music, books,
  documentaries, or food. For other times, email me to schedule an
  appointment
  \end{quote}
\item
  Requesting a 1-on-1, non-office-hour Zoom meeting:

  \begin{quote}
  If, after you've looked over the material yourself and reached out to
  classmates, you still need help but cannot attend office hours, you
  may email me to request a virtual Zoom meeting. For non-office-hour
  zoom meetings, I require at least a minimum of one-week's notice.
  \end{quote}
\item
  On-line communication

  \begin{quote}
  I require that all communication be done in a respectful, courteous,
  and professional manner. When in doubt, always err in these
  directions; for example, aim for being slightly more courteous than
  you may think you need to be. Nobody will object to an extra
  ``please'' or ``thank you.'' Always take a few minutes to read what
  you have written before posting a comment or sending an email -- pause
  and think -- could what you have written be construed as
  disrespectful, impolite, or snappy? When in doubt, re-think your words
  and tone, and re-calibrate your message. We may not always agree but
  disagreeing while simultaneously showing respect and courtesy toward
  someone is a skill that will serve you well throughout your life and
  in many contexts. Always make criticism constructive and not personal.
  If you disagree with someone, they should not feel personally
  attacked. Keep the ideas separate from the person. I will always keep
  these principles in mind when communicating with you -- please do the
  same toward me and your fellow students.
  \end{quote}
\end{itemize}

\textbf{I reserve the right to penalize individuals who do not
demonstrate these basic principles in their communications.}

\hypertarget{powerpoint-lecture-slides-availability}{%
\section{Powerpoint Lecture Slides
Availability:}\label{powerpoint-lecture-slides-availability}}

Please note that course is taught in a ``synchronous manner'' which
means that lectures will be delivered ``live''. \textbf{Each week the
PowerPoint lecture slides used in each class meeting will be posted on
Canvas (in the ``Modules'' section) on the following Friday.}

\hypertarget{use-of-lecture-material-powerpoint-lecture-slides-and-electronic-readings}{%
\subsection{Use of Lecture Material, PowerPoint Lecture Slides, and
Electronic
Readings:}\label{use-of-lecture-material-powerpoint-lecture-slides-and-electronic-readings}}

It is most important that you know that
\textcolor{red}{\bf{all} course materials (i.e, the course syllabus, assignments, handouts, exams, etc.) are the instructor’s property. Course readings, on the other hand, are the property of their writers and editors, and together, should be used only by students enrolled in this course. These materials should not be distributed or used beyond the member of this course. If you do, this will constitute a major legal violation with serious penalties.}
These materials are available to you to facilitate your own personal
learning in this class. Sale of course or reproduction of course
materials, including lecture content and information, are strictly
prohibited. All class material, particularly lectures, cannot be further
reproduced. Reproduction of audio or video of zoom lectures or
powerpoint slides by students or anyone else is \textbf{\emph{strictly
prohibited}} under any circumstances.
\textcolor{red}{\bf{Professor Jacinto reserves the right to take academic penalty measures and/or legal action if you are found in violation of any of the expectations stated here.}}

\hypertarget{creating-a-safe-and-inclusive-learning-environment}{%
\section{Creating a Safe and Inclusive Learning
Environment}\label{creating-a-safe-and-inclusive-learning-environment}}

\hypertarget{mandatory-reporting}{%
\subsection{Mandatory Reporting}\label{mandatory-reporting}}

As an instructor, one of my responsibilities is to help create a safe
learning environment on our campus. I also have a mandatory reporting
responsibility related to my role as an instructor. It is my goal that
you feel able to share information related to your life experiences in
classroom discussions, in your written work, and in our one-on-one
meetings. I will seek to keep information you share private to the
greatest extent possible. However, I am legally required to share
information regarding sexual misconduct with the University. Students
may speak to someone confidentially by contacting the
\href{https://www.csuchico.edu/counseling/}{Counseling and Wellness
Center} (530-898-6345) or
\href{https://www.csuchico.edu/safeplace/}{Safe Place} (530-898-3030).
Information on campus reporting obligations and other Title IX related
resources are available here: \url{https://www.csuchico.edu/title-ix/}

\hypertarget{statement-of-commitment-to-diversity-equity-and-inclusion}{%
\subsection{Statement of Commitment to Diversity, Equity, and
Inclusion}\label{statement-of-commitment-to-diversity-equity-and-inclusion}}

Education is transformative, and open intellectual inquiry is the
foundation of a university education and a democratic society. In the
spirit of shared humanity and concern for our community and world, the
faculty of Chico State's Sociology Department celebrate diversity as
central to our mission and affirm our solidarity with those individuals
and groups most at risk. In line with our departmental goals, we disavow
all racism, xenophobia, homophobia, sexism, Islamophobia, anti-Semitism,
classism, ableism, and hate speech or actions that attempt to silence,
threaten, and degrade others.

As educators, we affirm that language and texts, films and stories help
us to understand the experiences of others whose lives are different
from ours. We value critical reasoning, evidence-based arguments,
self-reflection, and the imagination. Building on these capacities, we
hope to inspire empathy, social and environmental justice, and an
ethical framework for our actions. We advocate for a diverse campus,
community, and nation inclusive of racial minorities, women, immigrants,
the LGBTQ+ community, and people of all religious faiths.

\hypertarget{statement-of-respect-for-diversity-in-learning-styles-perspectives-and-backgrounds}{%
\subsection{Statement of Respect for Diversity in Learning Styles,
Perspectives, and
Backgrounds}\label{statement-of-respect-for-diversity-in-learning-styles-perspectives-and-backgrounds}}

Each of us arrives in the classroom with a multitude of life
experiences, and we all learn in different ways. In designing this
class, my intent has been to provide many different methods to practice
and learn to support these different learning styles. In addition, to
learn about applying sociological perspectives to social issues, we will
use a variety of example to demonstrate their applicability. Some
examples will be simple and often use `made-up' scenarios, while others
deal with real life events and experiences about important social
issues. It is my hope that all materials in the class are respectful to
people from all backgrounds including diversity in sex/gender,
sexuality, disability, age, socioeconomic status, ethnicity/race,
immigration status, veteran status, religion, and culture. I invite your
feedback on this, and your suggestions are always welcome.

\hypertarget{undocumented-daca-or-ab-540-immigrant-students}{%
\subsection{Undocumented, DACA, or AB-540 Immigrant
Students}\label{undocumented-daca-or-ab-540-immigrant-students}}

As an educator, I strive to make courses accessible to all students
regardless of immigration status. If your status presents obstacles to
engaging in specific activities or fulfilling specific criteria, you may
request confidential accommodations. You may consult with the
\href{https://www.csuchico.edu/diversity/}{Office of Equity and
Diversity} or the \href{https://www.csuchico.edu/dreamcenter/}{Dream
Center} for examples of possible accommodations. Such arrangements will
not jeopardize your student status, your financial aid, or any other
part of your residence. Please advise me if and when you feel
comfortable during the semester so that I may make appropriate
alterations as needed.

\textbf{Note:} For all students addressing undocumented immigration as a
category of analysis in class, do not use the word ``illegal(s)'' in a
discussion. The term ``illegal(s)'' promotes a culture of intolerance
and violence toward foreign nationals and undocumented immigrants. A
more accurate and non-offensive term is ``undocumented immigrant(s).''
The use of this language signifies respect to the population addressed
and reflects our campus's most basic values of diversity and civility in
academic discourse.

\hypertarget{statement-on-eliminating-anti-blackness}{%
\subsection{Statement on Eliminating
Anti-Blackness}\label{statement-on-eliminating-anti-blackness}}

Faculty at CSU Chico strive to create an environment that supports
meaningful dialogue grounded in research, academic inquiry, and mutually
respectful relations. We also strive to remain conscious of and
attentive to the damage that anti-Blackness does to the lives of our
students, faculty, staff, administrators, and their related communities.
As such, faculty at CSU Chico denounce anti-Blackness and racial
violence in all forms and stand in solidarity with Black communities in
the fight for racial justice, equality, and equity. We pledge to remake
our institution as one that values, honors, and supports Black lives. We
recognize the impact of anti-Blackness on our students, and we invite
them to dialogue with their professors as we work to make our classrooms
anti-racist and dignity-affirming spaces.

\hypertarget{lgbtq-equality-statement}{%
\subsection{LGBTQ+ Equality Statement}\label{lgbtq-equality-statement}}

I am firmly committed to diversity and equality in all areas of campus
life, including specifically members of the LGBTQ+ community. In this
class I will work to promote an anti-discriminatory environment where
everyone feels safe and welcome. I recognize that discrimination can be
direct or indirect and take place at both institutional and personal
levels. I unequivocally believe that such discrimination is unacceptable
and I am committed to providing equality of opportunity for all by
eliminating any and all discrimination, harassment, bullying, or
victimization. The success of this policy relies on the support and
understanding of everyone in this class. We all have a responsibility
not to be offensive to each other, or to participate in, or condone
harassment or discrimination of any kind.

\hypertarget{active-duty-military-and-veterans-statement}{%
\subsection{Active Duty Military and Veterans
Statement}\label{active-duty-military-and-veterans-statement}}

If you are a veteran, on active duty, in the reserves, or a spouse or
dependent, and an aspect of your present or prior service makes it
difficult for you to fulfill the requirements of the course, keep me
informed about your situation and I will work with you. I realize that
you have little to no control over prompt benefits processing, VA
appointments, training schedules, deployments, and other aspects of
service. Campus resources for veterans, service members, and families
are located in the
\href{https://www.csuchico.edu/vets/index.shtml}{Veterans Education and
Transition Services (VETS) Office} in room 330 in the Student Services
Center (SSC) and is open Mon-Fri from 8:00AM to 4:45PM. The VETS Office
can be reached at (530) 898-5911 or
\href{mailto:VETS@csuchico.edu}{\nolinkurl{VETS@csuchico.edu}}.

\hypertarget{students-with-children}{%
\subsection{Students with children}\label{students-with-children}}

I understand that some students have children, and that your
responsibilities as a parent or caregiver are your top priority. While I
expect that you will make every effort to arrange suitable child care, I
also know that sometimes unexpected situations arise, and that child
care may be especially difficult to arrange right now. Therefore, I have
adopted the following policies:

\begin{itemize}
\item
  You are welcome to bring your child to my class as needed.
\item
  You are welcome to feed your child in my class, including
  breastfeeding.
\item
  If you bring your child to class, please provide quiet entertainment
  for them.
\item
  Please feel free to step out of class as needed to care for your
  child.
\item
  If you are struggling to manage your family and academic
  responsibilities, please meet with me so I can help.
\end{itemize}

\hypertarget{mental-health}{%
\subsection{Mental health}\label{mental-health}}

Being a university student is stressful and can negatively impact mental
health. The \href{https://www.csuchico.edu/counseling/}{WellCat
Counseling Center} offers free counseling services to regularly enrolled
students at California State University, Chico. There are a wide variety
of reasons students seek counseling:

\begin{itemize}
\tightlist
\item
  Resolving personal problems or conflicts.
\item
  Seeking clarification and support in making important life decisions.
\item
  Experiencing an emotional crisis due to a traumatic incident or a
  series of incidents.
\item
  Help in developing personal survival skills necessary to achieve
  personal goals.
\end{itemize}

\textcolor{ForestGreen}{\bf{Absences for mental health reasons are excused but the professor should be notified immediately (within 48 - 72 hours of absence) to prevent any penalty on your final grade.}}

\hypertarget{other-course-policies}{%
\section{Other Course Policies}\label{other-course-policies}}

\begin{itemize}
\item
  Unless you have a documented emergency, there will be no substitutions
  for assignments, quizzes, or papers. A missed assignment, quiz, or
  paper will result in a zero. Incomplete grades (or an ``I'' grade) are
  rare, and will be given at the discretion of the professor.
\item
  ``Pet Peeves'':

  \begin{enumerate}
  \def\labelenumi{\arabic{enumi}.}
  \tightlist
  \item
    Asking to be excused from rules that all classmates are following.
  \item
    \textbf{Inappropriate or disruptive behavior will not be tolerated.
    If this behavior persists, the student will be reported to the
    Dean's Office.}
  \end{enumerate}
\item
  As mentioned in the \protect\hyperlink{writing_assignments}{``Two (2)
  Critical Interpretation Essay Papers''} above, all papers must be
  12-point Times New Roman font (or 11-point Arial font), double-spaced,
  and have 1'' margins. Students' names are to be in the upper right
  hand corner with the date and student ID number included; also,
  include page numbers in the bottom right hand corner of the paper.
  Excessive spacing beyond this identification will result in the
  deduction of points. Finally, follow APA citation style, and give full
  references. For help with formatting citations and bibliographies,
  visit the ``Citing Sources'' page in Meriam Library's website:
  \url{https://libguides.csuchico.edu/cite}
\item
  When there is evidence that a student has committed plagiary, copied
  the work of others, allowed others to copy their work, cheated on a
  quiz, altered class material or scores, or has inappropriate
  possession of quizzes, or sensitive material, the incident will be
  investigated by Student Judicial Affairs.
  \textcolor{red}{\bf{The consequences for academic dishonesty are severe, including receiving an “F” in the course. No exceptions.}}
\item
  All written work will be assessed on content, depth, logic, and
  quality of written expression.
\item
  Students are responsible for handling the necessary paperwork for
  adding or dropping this class.
  \textcolor{red}{\bf{If a student does not withdraw, and does not attend class or complete required work, an “F” or “WU” will be reported for the final grade. Instructors may drop a student who does not attend the first two classes.}}
\end{itemize}

\hypertarget{university-policies-and-campus-resources}{%
\section{University Policies and Campus
Resources}\label{university-policies-and-campus-resources}}

\hypertarget{academic-integrity}{%
\subsection{Academic Integrity}\label{academic-integrity}}

Academic honesty and integrity are core values in this class and at
Chico State. Acts of plagiarism, dishonesty, and misrepresentation of
work are just a few examples of inappropriate behavior in this class.
Breeches in academic honesty and integrity will be submitted to the
Office of Student Conduct, Rights and Responsibilities for sanction.
Additionally, breaches in academic honesty will result in a failing
grade for the assignment to which the breach was committed, or, at the
discretion of the instructor, a failing grade for the course. For more
information about academic honesty and integrity, please visit
\url{http://www.csuchico.edu/prs/EMs/2004/04-036.shtml}.

\hypertarget{student-services}{%
\subsection{Student Services}\label{student-services}}

Student services are designed to assist students in the development of
their full academic potential and to motivate them to become
self-directed learners. Students can find support for services such as
skills assessment, individual or group tutorials, subject advising,
learning assistance, summer academic preparation and basic skills
development. Student services information can be found on the current
students page of the \href{https://www.csuchico.edu/sa/}{CSU Chico web
site}.

\hypertarget{student-learning-center}{%
\subsection{Student Learning Center}\label{student-learning-center}}

The mission of the Student Learning Center (SLC) is to provide services
that will assist CSU, Chico students to become independent learners. The
SLC prepares and supports students in their college course work by
offering a variety of programs and resources to meet student needs. The
SLC facilitates the academic transition and retention of students from
high schools and community colleges by providing study strategy
information, content subject tutoring, and supplemental instruction. I
suggest you visit the \href{https://www.csuchico.edu/slc/}{Student
Learning Center web site}.

\hypertarget{writing-center}{%
\subsection{Writing Center}\label{writing-center}}

Over the course of this semester, seek help at the Writing Center if you
would like additional guidance and critique. You can access the Writing
Center at
\href{https://www.csuchico.edu/slc/writing-center/index.shtml}{at their
website}. During graduate school, I found workshops and writing groups
at the Writing Center on campus to be helpful in building my motivation
and teaching me new strategies for writing efficiently. While it is not
a requirement, I nevertheless urge you to meet with a writing tutor to
review a writing assignment at least once during the semester.

\hypertarget{americans-with-disabilities-act}{%
\subsection{Americans with Disabilities
Act}\label{americans-with-disabilities-act}}

If you need course adaptations or accommodations because of a disability
or chronic illness, or if you need to make special arrangements in case
the building must be evacuated, please make an appointment with me as
soon as possible, or see me during office hours. Please also contact
Accessibility Resource Center (ARC) as they are the designated
department responsible for approving and coordinating reasonable
accommodations and services for students with disabilities. ARC will
help you understand your rights and responsibilities under the Americans
with Disabilities Act and provide you further assistance with requesting
and arranging accommodations. See contact info below:

\begin{quote}
\textbf{Accessibility Resource Center}\\
Student Services Center 170\\
530-898-5959\\
\href{mailto:arcdept@csuchico.edu}{\nolinkurl{arcdept@csuchico.edu}}\\
\url{https://www.csuchico.edu/arc/}
\end{quote}

\hypertarget{food-insecurity}{%
\subsection{Food insecurity}\label{food-insecurity}}

CSU Chico recognizes food and housing insecurity as a concern for many
people. Students who have difficulty securing adequate food or housing
have long and short term resources available at Chico State. Students
are encouraged to contact the
\href{https://www.csuchico.edu/chc/}{Center for Healthy Communities} or
530-898-5323 for assistance in applying for CalFresh benefits for long
term assistance with food access. For emergency food supplies, CalFresh
application help, and emergency housing information, students are
encouraged to visit the
\href{https://www.csuchico.edu/basic-needs/pantry.shtml}{Wildcat Food
Pantry}.

\hypertarget{final-thoughts}{%
\section{Final thoughts}\label{final-thoughts}}

This document is a roadmap for our semester. We will learn about global
problems and globalization as a social process. Like all your classes,
you will get out what you put into this course. Asking for help from one
another and your professors is important, don't be afraid to ask a
question about something you don't know or if you want to check your
knowledge about something you think you know. Do not wait to ask for
help! We are all here to learn; it is not a competition.

\textbf{If this document is updated, a copy will be supplied to you via
Canvas and changes will be announced in class.}

\newpage

\renewcommand{\arraystretch}{1.25}

\hypertarget{course-and-reading-schedule}{%
\section{Course and Reading
Schedule}\label{course-and-reading-schedule}}

The following is a tentative schedule for the Fall 2022 semester. It
lists readings and due dates, and \textbf{may be subject to change.} Any
changes will be announced in class and will be shown in Canvas
immediately. Use this information to schedule accordingly. If you miss
class, it is your responsibility to find out if any scheduling changes
were made in class.

Table \ref{tab:course_schedule} displays the course schedule for the
Fall 2022 semester. It lists the days and weeks we will meet for
lecture. \textbf{We will not meet during Labor Day (Sept.~05) and Fall
Break (Nov.~21-25).}

\begin{table}[!h]

\caption{\label{tab:course_schedule}Course Schedule.}
\centering
\fontsize{10}{12}\selectfont
\begin{tabular}[t]{llccc}
\toprule
\textbf{Date} & \textbf{Weekday} & \textbf{Class} & \textbf{Week} & \textbf{Lecture}\\
\midrule
2022-08-22 & Monday & 1 & 1 & 1\\
2022-08-24 & Wednesday & 2 & 1 & 2\\
2022-08-29 & Monday & 3 & 2 & 3\\
2022-08-31 & Wednesday & 4 & 2 & 4\\
2022-09-07 & Wednesday & 5 & 3 & 5\\
2022-09-12 & Monday & 6 & 4 & 6\\
2022-09-14 & Wednesday & 7 & 4 & 7\\
2022-09-19 & Monday & 8 & 5 & 8\\
2022-09-21 & Wednesday & 9 & 5 & 9\\
2022-09-26 & Monday & 10 & 6 & 10\\
2022-09-28 & Wednesday & 11 & 6 & 11\\
2022-10-03 & Monday & 12 & 7 & 12\\
2022-10-05 & Wednesday & 13 & 7 & 13\\
2022-10-10 & Monday & 14 & 8 & 14\\
2022-10-12 & Wednesday & 15 & 8 & 15\\
2022-10-17 & Monday & 16 & 9 & 16\\
2022-10-19 & Wednesday & 17 & 9 & 17\\
2022-10-24 & Monday & 18 & 10 & 18\\
2022-10-26 & Wednesday & 19 & 10 & 19\\
2022-10-31 & Monday & 20 & 11 & 20\\
2022-11-02 & Wednesday & 21 & 11 & 21\\
2022-11-07 & Monday & 22 & 12 & 22\\
2022-11-09 & Wednesday & 23 & 12 & 23\\
2022-11-14 & Monday & 24 & 13 & 24\\
2022-11-16 & Wednesday & 25 & 13 & 25\\
2022-11-28 & Monday & 26 & 15 & 26\\
2022-11-30 & Wednesday & 27 & 15 & 27\\
2022-12-05 & Monday & 28 & 16 & 28\\
2022-12-07 & Wednesday & 29 & 16 & 29\\
\bottomrule
\end{tabular}
\end{table}

\hypertarget{tentative-reading-and-assignments-schedule}{%
\subsection{Tentative Reading and Assignments
Schedule}\label{tentative-reading-and-assignments-schedule}}

Readings we will cover are listed below.
\textcolor{red}{\bf{Note that the professor reserves the right to change the readings and assignment due dates below based on new and developing information about what would best serve my students.}}
Changes in the syllabus will be highlighted in
\colorbox{Cyan}{\bf{cyan.}} In the event that a reading has not been
made available in a way that you can access it on Canvas or Perusall,
please email me immediately so the professor may repost the
reading/assignment in time for you to complete it before the next class
period. You will also be immediately notified when there the due date of
an assignment has changed.

\emph{These are questions you should consider as you complete each
reading:} What are the authors' main objectives/research questions? What
methods, if any, did the author use to explore this question? What are
the author's main conclusions? How do the readings build upon or
challenge other works we have read thus far? Answering these questions
on your own will better prepare you for quizzes, writing assignments,
and the paired presentation.

\hypertarget{course-schedule}{%
\section{\texorpdfstring{COURSE SCHEDULE
\textbf{\colorbox{YellowGreen}{(Syllabus last updated on August 09, 2022) }}}{COURSE SCHEDULE }}\label{course-schedule}}

\vspace{\baselineskip}

\begin{quote}
\hypertarget{introduction-theoretical-and-historical-foundations}{%
\subsection{INTRODUCTION: THEORETICAL AND HISTORICAL
FOUNDATIONS}\label{introduction-theoretical-and-historical-foundations}}
\end{quote}

\bigbreak

\hypertarget{week-01-0822---0824-a-first-look-at-contemporary-globalization}{%
\subsection{WEEK 01, 08/22 - 08/24: A First Look at Contemporary
Globalization}\label{week-01-0822---0824-a-first-look-at-contemporary-globalization}}

\textbf{Topics:}

\begin{enumerate}
\def\labelenumi{(\arabic{enumi})}
\tightlist
\item
  Some Core Trends: Inequality, Trade and Production
\item
  Trumpism \& Populism --- Outcomes of Neoliberal Globalization?
\end{enumerate}

\textbf{Required readings:}

\begin{itemize}
\item
  Milanovic B (2018). ``The Rise of the Global Middle Class and Global
  Plutocrats.'' In \emph{Global Inequality: A New Approach for the Age
  of Globalization}, 10-45. Harvard University Press, Cambridge, MA.
  ISBN 978-0-674-98403-5. \textbf{(Read pages 10-24, but 25-45 are
  \emph{recommended}).}
\item
  Stiglitz J (2019). ``The end of neoliberalism and the rebirth of
  history.'' \textless URL:
  \url{https://socialeurope.eu/the-end-of-neoliberalism-and-the-rebirth-of-history}\textgreater.
\item
  Piketty T (2016). ``We must rethink globalization, or Trumpism will
  prevail.'' \emph{The Guardian}. \textless URL:
  \url{https://bit.ly/3QgaV6m}\textgreater.
\end{itemize}

\textbf{Film (required):}

\begin{itemize}
\tightlist
\item
  \href{https://www.youtube.com/watch?v=9wjjQ55S4Nc}{\emph{The Realities
  of Trump's Trade War}} by Vice News (2019).
\end{itemize}

\textbf{Assignments Due:}

\begin{itemize}
\tightlist
\item
  Post a brief introduction of yourself on the ``Welcome Board''
\item
  \textbf{Syllabus Quiz must be completed by Friday August 26, by 11:59
  p.m.}
\end{itemize}

\bigbreak
\hrule
\newpage

\hypertarget{week-02-0829---0831-a-deeper-look.-globalization-and-the-capitalist-world-economy}{%
\subsection{WEEK 02, 08/29 - 08/31: A Deeper Look. Globalization and the
Capitalist
World-Economy}\label{week-02-0829---0831-a-deeper-look.-globalization-and-the-capitalist-world-economy}}

\textbf{Topics:}

\begin{enumerate}
\def\labelenumi{(\arabic{enumi})}
\tightlist
\item
  Globalization in the Long-Term: Capitalism as a World- Economy
\item
  Globalization as \emph{Capitalist Globalization}
\end{enumerate}

\textbf{Required Readings:}

\begin{itemize}
\item
  Wallerstein I (2004). ``The Modern World-System as a Capitalist World-
  Economy: Production, Surplus-Value, and Polarization.'' In
  \emph{World-Systems Analysis: An Introduction}. Duke University Press,
  Durham, NC.
\item
  Robinson W (2007). ``Beyond the Theory of Imperialism: Global
  Capitalism and the Transnational State.'' \emph{Societies Without
  Borders}, \emph{2}(1), 5-26. \textless URL: 2022-08-03\textgreater.
\end{itemize}

\textbf{Assignments Due:}

\begin{itemize}
\tightlist
\item
  Weekly Perusall Participation
\item
  \textbf{Quiz \#2 must be completed by Thursday September 1, by 11:59
  p.m.}
\end{itemize}

\bigbreak
\hrule

\begin{quote}
\begin{quote}
\hypertarget{part-i.-global-processes-accumulation-and-trade}{%
\subsection{PART I. GLOBAL PROCESSES: ACCUMULATION AND
TRADE}\label{part-i.-global-processes-accumulation-and-trade}}
\end{quote}
\end{quote}

\hypertarget{week-03-0907-accumulation-regimes-and-socio-economic-transformations-from-fordism-to-neoliberal-globalization-i}{%
\subsection{WEEK 03, 09/07: Accumulation Regimes and Socio-Economic
Transformations: From Fordism to Neoliberal Globalization,
I}\label{week-03-0907-accumulation-regimes-and-socio-economic-transformations-from-fordism-to-neoliberal-globalization-i}}

\colorbox{Melon}{\bf{No Class on Monday 09/05 (Labor Day)}}

\textbf{Topics:}

\begin{enumerate}
\def\labelenumi{(\arabic{enumi})}
\tightlist
\item
  Fordism, or, the So-Called Golden Age of Capitalism
\item
  Global Fordism, or, ``When Work Disappeared''
\end{enumerate}

\textbf{Required Readings:}

\begin{itemize}
\item
  Jessop B (1992). ``Fordism and Post-Fordism: A Critical
  Reformulation.'' In \emph{Pathways to Industrialization and Regional
  Development}, 1 edition, 42-62. Routledge, London, UK. ISBN
  978-0-203-99554-9.
\item
  Lipietz A (1984). ``Towards Global Fordism?'' \emph{New Left Review},
  33-47.
\end{itemize}

\textbf{Film (required):}

\begin{itemize}
\tightlist
\item
  Watch this short film on
  \href{https://www.youtube.com/watch?v=8PdmNbqtDdI}{``Fordism'' and
  ``Taylorism''}
\end{itemize}

\textbf{Assignments Due:}

\begin{itemize}
\tightlist
\item
  Weekly Perusall Participation
\end{itemize}

\bigbreak
\hrule

\hypertarget{week-04-0912---0914-accumulation-regimes-and-socio-economic-transformation-from-fordism-to-neoliberal-globalization-ii}{%
\subsection{WEEK 04, 09/12 - 09/14: Accumulation Regimes and
Socio-Economic Transformation: From Fordism to Neoliberal Globalization,
II}\label{week-04-0912---0914-accumulation-regimes-and-socio-economic-transformation-from-fordism-to-neoliberal-globalization-ii}}

\textbf{Topics:}

\begin{enumerate}
\def\labelenumi{(\arabic{enumi})}
\tightlist
\item
  On the Ruins of Fordism: The Rise of Neoliberalism
\end{enumerate}

\textbf{Required Readings:}

\begin{itemize}
\item
  Friedman M (1962). ``The Relation Between Economic Freedom and
  Political Freedom.'' In \emph{Capitalism and Freedom}, 7-21.
  University of Chicago Press, Chicago, IL.
\item
  Harvey D (2007). ``Freedom's Just Another Word.'' In \emph{Brief
  History of Neoliberalism}. Oxford University Press, New York.
\item
  Bonnefoy P (2017). ``Documenting U.S. Role in Democracy's Fall and
  Dictator's Rise in Chile.'' \emph{New York Times}, 3. \textless URL:
  \url{https://nyti.ms/3QfZ5Jl}\textgreater.
\end{itemize}

\textbf{Film (recommended):}

\begin{itemize}
\item
  \href{https://www.youtube.com/watch?v=B3B5qt6gsxY}{The Shock
  Doctrine}, directed by Mat Whitecross and Michael Winterbottom
  (Sundance Selects, 2009).
\item
  \href{https://www.freetochoosenetwork.org/programs/free_to_choose/index_90.php?id=the_power_of_the_market}{Free
  to Choose (PBS documentary by Milton Friedman)}. Watch 1990 version of
  the first episode available by going to Free to Choose Network. (You
  can skip the intro by Arnold Schwarzenegger by going to minute 6:46.)
\end{itemize}

\textbf{Assignments Due:}

\begin{itemize}
\tightlist
\item
  Weekly Perusall Participation
\item
  \textbf{Quiz \#2 must be completed by Thursday October 15, by 11:59
  p.m.}
\end{itemize}

\bigbreak
\hrule

\hypertarget{week-05-0919---0921-debating-free-trade-i-comparative-advantage-or-unequal-development}{%
\subsection{WEEK 05, 09/19 - 09/21: Debating Free Trade, I: Comparative
Advantage or Unequal
Development?}\label{week-05-0919---0921-debating-free-trade-i-comparative-advantage-or-unequal-development}}

\textbf{Topics:}

\begin{enumerate}
\def\labelenumi{(\arabic{enumi})}
\tightlist
\item
  Debating Free Trade in General
\end{enumerate}

\textbf{Required Readings:}

\begin{itemize}
\item
  Krugman P (2011). ``Increasing Returns in a Comparative Advantage
  World.'' In \emph{World Scientific Studies in International
  Economics}, volume 16, 43-51. WORLD SCIENTIFIC. ISBN 978-981-4340-36-6
  978-981-4340-37-3, doi: 10.1142/9789814340373\_0007 (URL:
  \url{https://doi.org/10.1142/9789814340373_0007}), \textless URL:
  \url{https://bit.ly/3Qk0m1U}\textgreater.
\item
  Amin S (1976). ``From Specialization to Dependence.'' In \emph{Unequal
  Development: An Essay on the Social Formations of Peripheral
  Capitalism}, 133-145. Harvester Press Limted, Sussex, England.
\end{itemize}

\textbf{Assignments Due:}

\begin{itemize}
\tightlist
\item
  Weekly Perusall Participation
\item
  \textbf{Critical Interpretation Essay \#1 due by Sunday September 25
  11:59 p.m.}
\end{itemize}

\bigbreak
\hrule

\hypertarget{week-06-0926---0928-debating-free-trade-ii-what-should-be-done-with-pollution--intensive-industries}{%
\subsection{WEEK 06, 09/26 - 09/28: Debating Free Trade, II: What Should
be Done with Pollution- Intensive
Industries?}\label{week-06-0926---0928-debating-free-trade-ii-what-should-be-done-with-pollution--intensive-industries}}

\textbf{Topics:}

\begin{itemize}
\tightlist
\item
  Debating Trade in Dirty Industries
\end{itemize}

\textbf{Required Readings:}

\begin{itemize}
\item
  Summers L (1991). ``Our Words: The Lawrence Summers Memo.''
  \textless URL: \url{https://bit.ly/3d8BUm9}\textgreater.
\item
  Johnson J, Pecquet G, Taylor L (2007). ``Potential Gains From Trade in
  Dirty Industries: Revisiting Lawrence Summers' Memo.'' \emph{Cato
  Journal}, \emph{27}(3), 397-410.
\item
  Baer HA (2012). ``Global Capitalism and Climate Change.'' In
  \emph{Handbook on International Political Economy}, 395-414. World
  Scientific, Singapore.
\end{itemize}

\textbf{Assignments Due:}

\begin{itemize}
\tightlist
\item
  Weekly Perusall Participation
\end{itemize}

\bigbreak
\hrule

\hypertarget{week-07-1003---1005-debating-free-trade-iii-unearthing-the-sources-of-global-unequal-exchange}{%
\subsection{WEEK 07, 10/03 - 10/05: Debating Free Trade, III: Unearthing
the Sources of Global Unequal
Exchange}\label{week-07-1003---1005-debating-free-trade-iii-unearthing-the-sources-of-global-unequal-exchange}}

\textbf{Topics:}

\begin{enumerate}
\def\labelenumi{(\arabic{enumi})}
\tightlist
\item
  A Historical Perspective on Global Trade: Imperialism and Unequal
  Exchange
\item
  Midterm Review on \emph{Wednesday October 5}
\end{enumerate}

\textbf{Reading Readings:}

\begin{itemize}
\tightlist
\item
  Clark B, Foster JB (2009). ``Ecological Imperialism and the Global
  Metabolic Rift: Unequal Exchange and the Guano/Nitrates Trade.''
  \emph{International Journal of Comparative Sociology}, \emph{50}(3-4),
  311-334. ISSN 0020-7152, 1745-2554, doi: 10.1177/0020715209105144
  (URL: \url{https://doi.org/10.1177/0020715209105144}), \textless URL:
  \url{https://bit.ly/3zzNLkJ}\textgreater.
\end{itemize}

\textbf{Assignments Due:}

\begin{itemize}
\tightlist
\item
  Weekly Perusall Participation
\item
  \textbf{Midterm Exam must be completed by Thursday October 6, by 11:59
  p.m.}
\end{itemize}

\bigbreak
\hrule

\begin{quote}
\begin{quote}
\hypertarget{part-ii.-dimensions-of-globalization}{%
\subsection{PART II. DIMENSIONS OF
GLOBALIZATION}\label{part-ii.-dimensions-of-globalization}}
\end{quote}
\end{quote}

\hypertarget{week-08-1010---1012-global-labor-forces}{%
\subsection{WEEK 08, 10/10 - 10/12: Global Labor
Forces}\label{week-08-1010---1012-global-labor-forces}}

\textbf{Topics:}

\begin{enumerate}
\def\labelenumi{(\arabic{enumi})}
\tightlist
\item
  Racialization of Global Labor Force
\item
  Global Labor Struggles
\end{enumerate}

\textbf{Required Readings:}

\begin{itemize}
\item
  Bonacich E, Alimahomed S, Wilson JB (2008). ``The Racialization of
  Global Labor.'' \emph{American Behavioral Scientist}, \emph{52}(3),
  342-355. \textless URL: 2022-08-04\textgreater.
\item
  Silver BJ, Karatasli SS (2015). ``Historical Dynamics of Capitalism
  and Labor Movements.'' In \emph{The Oxford Handbook of Social
  Movements}, 133-145. Oxford University Press, United Kingdom.
\end{itemize}

\textbf{Recommended Readings:} + Wallerstein I (1988). ``The Ideological
Tensions of Capitalism: Universalism Versus Racism and Sexism.'' In
\emph{Race, Nation, Class: Ambiguous Identities}, 29-36. Verso.

\textbf{Assignments Due:}

\begin{itemize}
\tightlist
\item
  Weekly Perusall Participation
\item
  Choose partner and topic for \emph{paired presentation} assignment by
  \textbf{the end of Wednesday's class}
\end{itemize}

\hypertarget{week-09-1017---1019-gendered-global-accumulation}{%
\subsection{WEEK 09, 10/17 - 10/19: Gendered Global
Accumulation}\label{week-09-1017---1019-gendered-global-accumulation}}

\textbf{Topics:}

\begin{enumerate}
\def\labelenumi{(\arabic{enumi})}
\tightlist
\item
  Patriarchy and Global Accumulation
\item
  Complex Gender Configurations in Globalization
\end{enumerate}

\textbf{Required Readings:}

\begin{itemize}
\item
  Mies M (2007). ``Patriarchy and accumulation on a world scale
  revisited. (Keynote lecture at the Green Economics Institute, Reading,
  29 October 2005).'' \emph{International Journal of Green Economics},
  \emph{1}(3/4), 268. ISSN 1744-9928, 1744-9936, \textless URL:
  \url{http://www.inderscience.com/link.php?id=13059}\textgreater.
\item
  Chen MA, Vanek J, Lund F, Heintz J (2005). ``Chapter 2: The Totality
  of Women's Work.'' In \emph{Progress of the world's women 2005: women,
  work, \& poverty}, 22-35. United Nations Development Fund for Women,
  New York, N.Y.
\item
  George S (2000). ````Dirty Nurses'' and ``Men Who Play''.'' In Buroway
  M, Blum JA, George S, Gille Z, Gowan T, Haney L, Klawiter M, Lopez SH,
  O'Riain S, Thayer M (eds.), \emph{Global Ethnography: Forces,
  Connections, and Imaginations in a Postmodern World}, 1 edition,
  144-174. University of California Press, Berkeley, CA. ISBN
  0-520-22216-4.
\end{itemize}

\textbf{Assignments Due:}

\begin{itemize}
\tightlist
\item
  Weekly Perusall Participation
\item
  \textbf{Quiz \#3 must be completed by Thursday October 20, by 11:59
  p.m.}
\end{itemize}

\bigbreak
\hrule

\hypertarget{week-10-1024---1026-global-environmental-injustice}{%
\subsection{WEEK 10, 10/24 - 10/26: Global Environmental
(In)justice}\label{week-10-1024---1026-global-environmental-injustice}}

\textbf{Topics:}

\begin{enumerate}
\def\labelenumi{(\arabic{enumi})}
\tightlist
\item
  Environmental Racism
\item
  Climate Change and Environmental Injustice
\end{enumerate}

\textbf{Required Readings:}

\begin{itemize}
\item
  Bullard RD (1994). ``Environmental Racism and `Invisible'
  Communities.'' \emph{West Virginia Law Review}, \emph{96}(4),
  1037-1050.
\item
  Timmons Roberts J, Parks BC (2007). ``Fueling Injustice:
  Globalization, Ecologically Unequal Exchange and Climate Change.''
  \emph{Globalizations}, \emph{4}(2), 193-210. ISSN 1474-7731,
  1474-774X, doi: 10.1080/14747730701345218 (URL:
  \url{https://doi.org/10.1080/14747730701345218}), \textless URL:
  \url{http://www.tandfonline.com/doi/abs/10.1080/14747730701345218}\textgreater.
\end{itemize}

\textbf{Films (required):}

\begin{itemize}
\item
  \emph{TeleSUR's}
  \href{https://www.youtube.com/watch?v=cb4PH7ak_iU}{``Empire Files: The
  Sacrifice Zones of Hurricane Harvey''}
\item
  \emph{The Intercept's}
  \href{https://www.youtube.com/watch?v=pTiZtYaB3Zo}{``The Battle for
  Paradise: Naomi Klein Reports from Puerto Rico''}
\end{itemize}

\textbf{Assignments Due:}

\begin{itemize}
\tightlist
\item
  Weekly Perusall Participation
\item
  \textbf{Critical Interpretation Essay \#2 must be turned in by Friday
  October 28, by 11:59 p.m.}
\end{itemize}

\bigbreak

\hypertarget{week-11-1031---1102-global-health-injustice}{%
\subsection{WEEK 11, 10/31 - 11/02: Global Health
(In)justice}\label{week-11-1031---1102-global-health-injustice}}

\textbf{Topics:}

\begin{enumerate}
\def\labelenumi{(\arabic{enumi})}
\tightlist
\item
  Globalization and Pandemics I: Ebola
\item
  Globalization and Pandemics II: Coronavirus (Covid-19)
\end{enumerate}

\textbf{Required Readings:}

\begin{itemize}
\item
  Wallace R, Wallace R (2016). ``Ebola's Ecologies.'' \emph{New Left
  Review}, 1-13.
\item
  Wallace R, Liebman A, Chaves LF, Wallace R (2020). ``COVID-19 and
  Circuits of Capital.'' \emph{Monthly Review}, 1-15. ISSN 0027-0520,
  0027-0520, doi: 10.14452/MR-072-01-2020-05\_1 (URL:
  \url{https://doi.org/10.14452/MR-072-01-2020-05_1}), \textless URL:
  \url{https://monthlyreviewarchives.org/index.php/mr/article/view/MR-072-01-2020-05_1}\textgreater.
\end{itemize}

\textbf{Films (required):}

\begin{itemize}
\tightlist
\item
  \href{https://www.pbs.org/wgbh/frontline/documentary/ebola-outbreak/}{\emph{Ebola
  Outbreak}} (Boston: WGBH/Frontline, 2014)
\item
  \href{https://www.youtube.com/watch?v=TPpoJGYlW54}{\emph{How Wildlife
  Trade is Linked to Coronavirus}} by Vox (2020)
\end{itemize}

\textbf{Assignments Due:}

\begin{itemize}
\tightlist
\item
  Weekly Perusall Participation
\end{itemize}

\bigbreak
\hrule

\begin{quote}
\begin{quote}
\hypertarget{part-iii.-concrete-analysis-of-concrete-situations}{%
\subsection{PART III. CONCRETE ANALYSIS OF CONCRETE
SITUATIONS}\label{part-iii.-concrete-analysis-of-concrete-situations}}
\end{quote}
\end{quote}

\hypertarget{week-12-1107---1109-latin-america-in-capitalist-globalization-processes-of-incorporation}{%
\subsection{WEEK 12, 11/07 - 11/09: Latin America in Capitalist
Globalization: Processes of
Incorporation}\label{week-12-1107---1109-latin-america-in-capitalist-globalization-processes-of-incorporation}}

\textbf{Topics:}

\begin{enumerate}
\def\labelenumi{(\arabic{enumi})}
\tightlist
\item
  Latin America in Globalization: Socio-Economic Transformation and
  Migration
\item
  Latin America in Globalization: Informalization, Feminization and
  Immiseration
\end{enumerate}

\textbf{Required Readings:}

\begin{itemize}
\item
  Robinson WI (2012). ``Latin America in the New Global Capitalism.''
  \emph{NACLA Report on the Americas}, \emph{45}(2), 13-18.
\item
  Robinson WI, Santos X (2014). ``Global Capitalism, Immigrant Labor,
  and the Struggle for Justice.'' \emph{Class Race Corporate Power},
  \emph{2}(3). \textless URL: 2022-08-05\textgreater.
\item
  Robinson WI (2008). ``Chapter 5: The Antinomies of Global Capitalism
  and the Twilight of Neoliberalism.'' In \emph{Latin America and Global
  Capitalism: A Critical Globalization Perspective}, 226-286. John
  Hopkins University, Baltimore.
  \colorbox{Goldenrod}{\bf{Read pp. 241-244 and 247-254 only}}
\end{itemize}

\textbf{Assignments Due:}

\begin{itemize}
\tightlist
\item
  Weekly Perusall Participation
\item
  \textbf{Quiz \#3 must be completed by Thursday October 20, by 11:59
  p.m.}
\end{itemize}

\bigbreak
\hrule

\hypertarget{week-13-1114---1116-latin-america-in-capitalist-globalization-processes-of-state-led-resistance}{%
\subsection{WEEK 13, 11/14 - 11/16: Latin America in Capitalist
Globalization: Processes of State-led
Resistance}\label{week-13-1114---1116-latin-america-in-capitalist-globalization-processes-of-state-led-resistance}}

\textbf{Topics:}

\begin{enumerate}
\def\labelenumi{(\arabic{enumi})}
\tightlist
\item
  Cuba's Arduous Resistance to Capitalist Globalization
\item
  \emph{The Pink Tide}: A Latin American Turn to the Left
\end{enumerate}

\textbf{Required Readings:}

\begin{itemize}
\item
  Yaffe H (2020). ``Surviving the Crisis: The Special Period.'' In
  \emph{We are Cuba! how a revolutionary people have survived in a
  post-Soviet world}, 37-68. Yale University Press, New Haven.
\item
  Bull B (2013). ``Social Movements and the `Pink Tide' Governments in
  Latin America: Transformation, Inclusion and Rejection.'' In Stokke K,
  Törnquist O (eds.), \emph{Democratization in the Global South}, 75-99.
  Palgrave Macmillan UK, London. \textless URL: 2022-08-05\textgreater.
\item
  Robinson WI (2008). ``Chapter 6: A New Cycle of Resistance: The Future
  of Latin America and Global Society.'' In \emph{Latin America and
  Global Capitalism: A Critical Globalization Perspective}, 287-359.
  John Hopkins University, Baltimore.

  \colorbox{Goldenrod}{\bf{Read the section "Venezuela and the Bolivarian Revolution" (pp. 322-326) only}}
\item
  Weisbrot M, Sachs J (2019). ``Economic Sanctions as Collective
  Punishment: The Case of Venezuela.''
\end{itemize}

\bigbreak
\hrule

\hypertarget{week-14-1121---1123-fall-break---no-classes-will-be-held}{%
\subsection{WEEK 14, 11/21 - 11/23: Fall Break - No classes will be
held}\label{week-14-1121---1123-fall-break---no-classes-will-be-held}}

\begin{itemize}
\tightlist
\item
  \textcolor{Magenta}{\bf{No required readings or assignments due this week.}}
\end{itemize}

\bigbreak
\hrule

\hypertarget{week-15-1128---1130-debating-chinas-place-in-globalization}{%
\subsection{WEEK 15, 11/28 - 11/30: Debating China's Place in
Globalization}\label{week-15-1128---1130-debating-chinas-place-in-globalization}}

\textbf{Topics:}

\begin{enumerate}
\def\labelenumi{(\arabic{enumi})}
\tightlist
\item
  China's Rise as Historic East Asian Renaissance
\item
  China's Rise as the PRC's Road to Capitalism
\end{enumerate}

\textbf{Required Readings:}

\begin{itemize}
\item
  Arrighi G (2007). ``Origins and Dynamic of the Chinese Ascent.'' In
  \emph{Adam Smith in Beijing: lineages of the twenty-first century},
  351-378. Verso, London ; New York.
\item
  Andreas J (2008). ``Changing Colours in China.'' \emph{New Left
  Review}, \emph{5}, 123-142.
\end{itemize}

\textbf{Assignments Due:}

\begin{itemize}
\tightlist
\item
  Weekly Perusall Participation
\item
  \textbf{Quiz \#4 must be completed by Thursday December 1, by 11:59
  p.m.}
\end{itemize}

\bigbreak
\hrule

\hypertarget{week-16-1205---1207-dynamics-of-global-resistance}{%
\subsection{WEEK 16, 12/05 - 12/07: Dynamics of Global
Resistance}\label{week-16-1205---1207-dynamics-of-global-resistance}}

\begin{itemize}
\item
  Wallerstein I (2002). ``New Revolts Against the System.'' \emph{New
  Left Review}, \emph{18}, 29-39.
\item
  Zizek S (2007). ``Resistance is Surrender.'' \emph{London Review of
  Books}.
\item
  Wright EO (2006). ``Two redistributive proposals---universal basic
  income and stakeholder grants.'' \emph{Focus}, \emph{24}(2), 5-7.
\item
  Noisecat JB (2016). ``The Indigenous Revolution.'' \emph{The Jacobin}.
\end{itemize}

\textbf{Assignments Due:}

\begin{itemize}
\tightlist
\item
  Weekly Perusall Participation
\item
  \textbf{Group presentation/video due Friday December 9, by 11:59 p.m.}
\item
  \textbf{Extra credit assignments due Friday December 9, by 11:59 p.m.}
\item
  \textbf{Re-write of lowest scored paper must be turned in by Friday
  December 9, by 11:59 p.m.}
\end{itemize}

\hypertarget{final-exam-will-be-held-on-tuesday-december-13.-more-details-will-be-posted-on-canvas.}{%
\subsection{Final Exam will be held on Tuesday December 13. More details
will be posted on
Canvas.}\label{final-exam-will-be-held-on-tuesday-december-13.-more-details-will-be-posted-on-canvas.}}

\bigbreak
\hrule




\end{document}

\makeatletter
\def\@maketitle{%
  \newpage
%  \null
%  \vskip 2em%
%  \begin{center}%
  \let \footnote \thanks
    {\fontsize{18}{20}\selectfont\raggedright  \setlength{\parindent}{0pt} \@title \par}%
}
%\fi
\makeatother